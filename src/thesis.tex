%--------------------------------------------------------------------%
%
% Berkas utama templat LaTeX.
%
% author Petra Barus, Peb Ruswono Aryan
%
%--------------------------------------------------------------------%
%
% Berkas ini berisi struktur utama dokumen LaTeX yang akan dibuat.
%
%--------------------------------------------------------------------%

\documentclass[12pt, a4paper, onecolumn, oneside, final]{report}

\usepackage{enumitem}
\usepackage{float}

% Default fixed font does not support bold face
\DeclareFixedFont{\ttb}{T1}{txtt}{bx}{n}{12} % for bold
\DeclareFixedFont{\ttm}{T1}{txtt}{m}{n}{12}  % for normal

% custom colors
\usepackage{color}
\definecolor{deepblue}{rgb}{0,0,0.5}
\definecolor{deepred}{rgb}{0.6,0,0}
\definecolor{deepgreen}{rgb}{0,0.5,0}

\usepackage{listings}
\usepackage{amsmath}
\newcommand\numberthis{\addtocounter{equation}{1}\tag{\theequation}}

\usepackage[final]{hyperref} % adds hyper links inside the generated pdf file
\hypersetup{
	colorlinks=true,       % false: boxed links; true: colored links
	linkcolor=blue,        % color of internal links
	citecolor=blue,        % color of links to bibliography
	filecolor=magenta,     % color of file links
	urlcolor=blue         
}

% Python style for highlighting
\newcommand\pythonstyle{\lstset{
language=Python,
basicstyle=\ttm,
otherkeywords={}, % tambah disini buat highlight biru (fungsi internal python: def, print, dll)
keywordstyle=\ttb\color{deepblue},
emph={},          % buat highlight merah (gatau buat apa ya wkwk, kalo mau isi nama2 library paling)
emphstyle=\ttb\color{deepred},    % Custom highlighting style
stringstyle=\color{deepgreen},
frame=tb,                         % Any extra options here
showstringspaces=false            % 
}}

% Python environment
\lstnewenvironment{python}[1][]
{
\pythonstyle
\lstset{#1}
}
{}

% Python for inline
\newcommand\pythoninline[1]{{\pythonstyle\lstinline!#1!}}


\input{config/if-itb-thesis.sty}

\makeatletter

\makeatother

\bibliography{references}

\begin{document}

    %Basic configuration
    \title{Analisis Metode LSTM Autoencoder dan Bayesian Probability untuk Anomaly Detection}
    \date{}
    \author{
        Saleh Zaidan \quad 13318006 \\
        Farrel Dzaudan Naufal \quad 13318048
    }

    \pagenumbering{roman}
    \setcounter{page}{1}

    \clearpage
\pagestyle{empty}

\begin{center}
\smallskip

    \Large \bfseries \MakeUppercase{\thetitle}
    \vfill

    \Large Laporan Tugas Besar
    \vfill

    %\large Disusun sebagai syarat kelulusan tingkat sarjana
    %\vfill

    \large Oleh

    \Large \theauthor

    \vfill
    \begin{figure}[h]
        \centering
      	\includegraphics[width=0.25\textwidth]{resources/cover-ganesha.jpg}
    \end{figure}
    \vfill

    \large
    \uppercase{
        Program Studi Teknik Fisika \\
        Fakultas Teknologi Industri \\
        Institut Teknologi Bandung
    }

    2020

\end{center}

\clearpage

    %\input{chapters/approval}
    %\input{chapters/statement}

    \pagestyle{plain}

    %\input{chapters/abstract-id}
    %\input{chapters/abstract-en}
    %\input{chapters/forewords}

    \titleformat*{\section}{\centering\bfseries\Large\MakeUpperCase}

    \tableofcontents
    %\listoffigures
    %\listoftables

    \titleformat*{\section}{\bfseries\Large}
    \pagenumbering{arabic}

    %----------------------------------------------------------------%
    % Konfigurasi Bab
    %----------------------------------------------------------------%
    \setcounter{page}{1}
    \renewcommand{\chaptername}{BAB}
    \renewcommand{\thechapter}{\Roman{chapter}}
    %----------------------------------------------------------------%

    %----------------------------------------------------------------%
    % Dafter Bab
    % Untuk menambahkan daftar bab, buat berkas bab misalnya `chapter-6` di direktori `chapters`, dan masukkan ke sini.
    %----------------------------------------------------------------%
    \chapter{Pendahuluan}

\section{Prelude: Preventative Maintenance for Machine Failure}

\section{Anomaly Detection: An Unsupervised Learning}

    \chapter{Metode Acuan}

\section{What Data is it?}

Data yang digunakan adalah data pembacaan sensor pada pompa air yang berasal dari seorang pengguna di Kaggle. Pompa air yang digunakan sering mengalami kerusakan sehingga menimbulkan permasalahan serius terhadap kehidupan warga. Tim yang bekerja tidak dapat melihat pola apapun pada data ketika sistem rusak, sehingga diharapkan dapat ditemukan suatu cara untuk memprediksinya.

% Perujukan literatur dapat dilakukan dengan menambahkan entri baru di berkas. Tulisan ini merujuk pada \parencite{knuth2001art}

% \begin{figure}[h]
%     \centering
%     \includegraphics[width=0.8\textwidth]{resources/chapter-2-infrastructure-diagram.png}
%     \caption{Contoh gambar}
% \end{figure}

\section{Data Preprocessing}

Data terdiri dari 220.320 entri yang terbagi dalam 3 grup utama:

\begin{enumerate}
    \item Timestamp
    \item Sensor
    \item Status mesin: Label target yang diprediksi ketika kerusakan akan terjadi
\end{enumerate}

Timestamp tertulis dalam format: \texttt{tahun-bulan-tanggal jam:menit:detik}

Sensor tersebar dalam 52 kolom dari \texttt{sensor\_00} hingga \texttt{sensor\_51} yang terdiri dari nilai float.

Status mesin terdiri dari 3 kelas: NORMAL, BROKEN, dan RECOVERING yang masing-masing menyatakan kondisi normal, rusak, dan pulih pada pompa.

    \subsection{Data Cleaning}

    Terdapat beberapa hal yang harus dilakukan terlebih dahulu pada data:

    \begin{enumerate}
        \item Menghapus kolom berlebih
        \item Menghapus duplikat
        \item Mengatasi nilai yang hilang (not available/NA)
        \item Mengkonversi tipe data ke tipe yang benar
    \end{enumerate}
    
    Untuk mengatasi nilai hilang dicari 10 sensor dengan jumlah NA terbanyak. Kemudian NA pada sensor tersebut diisi oleh nilai mean-nya (impute), sedangkan NA pada sensor lain dihapus.

    \subsection{Dimensional Reduction}

    Komputasi dengan seluruh kolom dari sensor akan membutuhkan waktu yang lama dan tidak efisien. Sehingga diterapkan principal component analysis (PCA) untuk menghasilkan fitur baru yang dapat digunakan pada modeling. Namun sebelum itu data perlu diskala terlebih dahulu karena PCA merupakan algoritma berbasis jarak. Jika dilihat pada 10 data pertama besar nilai bacaan tidak konsisten pada tiap sensor, beberapa sangat besar dan yang lain justru sangat kecil.

        \subsubsection{Principal Component Analysis}

        Setelah dianalisis berdasarkan nilai variansi ternyata 2 komponen utama pertama adalah yang paling penting, sehingga PCA dilakukan dengan menggunakan 2 komponen tersebut yang kemudian digunakan pada pelatihan model.

        \subsubsection{Stationarity and Autocorrelation}

        Stationarity merupakan perilaku data time-series saat nilai mean dan standar deviasinya tidak berubah sepanjang waktu. Sedangkan autocorrelation merupakan perilaku data ketika terkorelasi dengan dirinya sendiri pada periode waktu yang berbeda.

        Untuk mengecek stationarity secara kuantitatif dilakukan Augmented Dickey-Fuller Test pada kedua komponen utama dari PCA.

        Untuk mengecek autocorrelation digunakan plot ACF (autocorrelation function) untuk memperoleh hasil secara visual.

\section{Modeling}

    \subsection{Interquartile Range (IQR)}

    Strategi dari model ini adalah:

    \begin{enumerate}
        \item Hitung nilai IQR, yaitu beda dari persentil ke-75 (Q3) dan ke-25 (Q1)
        \item Hitung batas atas dan bawah untuk outlier
        \item Filter data yang berada di atas dan di bawah batas dan tandai sebagai outlier
        \item Plot outlier di atas data time-series
    \end{enumerate}

    \subsection{K-Means Clustering}

    Strategi dari model ini adalah:

    \begin{enumerate}
        \item Hitung jarak antara tiap titik dan centroid terdekatnya. Jarak terbesar dianggap sebagai anomali.
        \item Gunakan parameter \texttt{outliers\_fraction} untuk memberi informasi pada algoritma tentang proporsi outlier yang ada pada data. Situasi dapat berbeda untuk tiap dataset. Namun sebagai langkah awal asumsikan \texttt{outliers\_fraction} = 0.13 (13\% dari data total adalah outlier).
        \item Hitung jumlah outlier dengan \texttt{outliers\_fraction}.
        \item Atur threshold sebagai jarak minimum dari outlier.
        \item Filter data dengan jarak lebih dari threshold sebagai anomali.
        \item Visualisasi anomali pada plot time-series.
    \end{enumerate}

    \subsection{Isolation Forest}

    Model Isolation Forest ‘mengisolasi’ observasi dengan memilih fitur secara acak dan memilih nilai split antara nilai maksimum dan minimum untuk fitur terkait.

    \section{Evaluation}

    Dari ketiga model diperoleh jumlah data normal dan data anomali sebagai berikut.

    \begin{table}[h]
        \centering
        \begin{tabular}{|l|r|r|}
            \hline
            \multicolumn{1}{|c|}{\textbf{Metode}} & \multicolumn{1}{c|}{\textbf{Normal}} & \multicolumn{1}{c|}{\textbf{Anomali}} \\ \hline
            IQR                & 189.644 & 29.877 \\ \hline
            K-Means Clustering & 190.984 & 28.537 \\ \hline
            Isolation Forest   & 190.983 & 28.538 \\ \hline
        \end{tabular}
    \end{table}
    \chapter{Metode Kembangan}

Terdapat 2 metode yang penulis kembangkan. Metode pertama adalah model LSTM Autoencoder yang merupakan metode umum digunakan data engineer untuk anomaly detection. Metode kedua adalah model Bayesian Probability yang merupakan model inovasi buatan penulis sendiri.

\section{LSTM Autoencoder}

Long-short term memory (LSTM) neural network adalah arsitektur jenis khusus dari recurrent neural network (RNN) yang dapat mengingat informasi dalam jangka waktu yang panjang.

LSTM cocok untuk mengklasifikasi, memproses, dan memprediksi data time series, karena mungkin terdapat jeda waktu yang tidak diketahui di antara peristiwa penting pada time series.

\begin{figure}[h]
    \centering
    \includegraphics[width=0.8\textwidth]{resources/LSTM_cell.png}
    \caption{Satu sel LSTM}
\end{figure}

LSTM Autoencoder adalah model neural network yang menggunakan LSTM sebagai hidden layernya untuk belajar menyalin input ke output. Autoencoder terdiri dari dua bagian, yaitu encoder yang menyalin input menjadi kode dan decoder yang menyalin kode menjadi output.

\begin{figure}[h]
    \centering
    \includegraphics[width=0.8\textwidth]{resources/lstm_ae.png}
    \caption{LSTM Autoencoder}
\end{figure}

Dengan begitu autoencoder dapat mempelajari fitur-fitur yang dimiliki input sehingga dapat melakukan prediksi, dan jika menggunakan LSTM sebagai hidden layernya maka dapat digunakan untuk memprediksi input berupa time series.

\section{Bayesian Probability}
\subsection{Idea and Concept}

Ide awal untuk menggunakan model Bayesian untuk anomaly detection berasal dari konsep anomaly detection menggunakan pendekatan identifikasi data outlier. Beberapa metode umum untuk mendeteksi data outlier seperti K-Means Clustering dan Isolation Forest menggunakan konsep pengelompokkan data atau clustering data. Sehingga akan diperoleh data-data yang berada didalam cluster tersebut dan data yang berada diluar cluster diidentifikasi sebagai outlier.

Terlihat disini bahwa pendekatan clustering ini mirip dengan model Bayesian untuk respon biner. Asumsikan setiap data memiliki suatu probabilitas apakah data tersebut termasuk pada cluster normal ataukah outlier. Dengan begitu, dapat diekspektasi distribusi probabilitas akan memiliki kontur lingkaran. Semakin dekat suatu data dengan pusat lingkaran, maka data tersebut semakin likely merupakan data normal. Pusat lingkaran akan menjadi titik dengan probabilitas suatu data merupakan data normal sebesar 1, artinya pusat lingkaran adalah titik data ideal dari kondisi normal dataset, dalam kasus ini yaitu hasil pengukuran ideal dari sensor-sensor pada sistem mesin. Sebaliknya, semakin jauh dari pusat lingkaran maka akan semakin unlikely data tersebut adalah data normal, atau dalam arti lain: data tersebut semakin likely merupakan anomaly.

\subsection{Deriving the Maths}

Definisikan data yang menjadi input model ini adalah 2 komponen utama dari hasil Principal Componen Analysis (PCA). Hal ini karena semakin tinggi dimensi data input, model akan semakin kompleks. Oleh karena itu data yang akan digunakan adalah komponen pertama PCA (pc1) sebagai $x_0$ dan komponen kedua PCA (pc2) sebagai $x_1$.

\begin{equation}
    \mathbf{x_n}=\begin{bmatrix} x_0 \\ x_1 \end{bmatrix}
\end{equation}

Sesuai ekspektasi awal, distribusi probabilitas akan memiliki kontur lingkaran. Maka, diperlukan 2 buah fungsi. Pertama adalah fungsi yang memetakan $\mathbf{x_n}$ ke suatu nilai dengan kontur lingkaran. Kedua adalah fungsi yang memetakan nilai kontur tersebut ke dalam range $[0,1]$, karena probabilitas hanya bernilai diantara 0 hingga 1.

Definisikan fungsi pertama sebagai
\begin{equation}
    f(\mathbf{w,x_n})=(w_0+x_0)^2+(w_1+x_1)^2 \label{fungsi_f}
\end{equation}
dengan
\begin{equation}
    \mathbf{w}=\begin{bmatrix} w_0 \\ w_1 \end{bmatrix}
\end{equation}
sebagai vektor parameter yang menentukan bentuk dan posisi kontur. Kemudian, dengan menetapkan data normal memiliki respon $t_n=1$ dan data anomaly memiliki respon $t_n=0$, didefinisikan fungsi kedua sebagai
\begin{equation*}
    P(T_n=1|\mathbf{x_n}, \mathbf{w}) = \frac{1}{1+f(\mathbf{w,x_n})}
\end{equation*}
Untuk penyederhanaan penulisan, $f(\mathbf{w,x_n})$ akan disingkat sebagai $f$ dan $P(T_n=1|\mathbf{x_n}, \mathbf{w})$ sebagai $P_n$.
\begin{equation}
    P_n = \frac{1}{1+f}
\end{equation}
Konsep model bayesian menggunakan pendekatan dengan memisalkan suatu sampling dengan rasio sukses $p = P(T_n=1|\mathbf{x_n},\mathbf{w})$, sampling dilakukan hanya 1 kali ($N=1$) untuk total sukses sebanyak $Y_N=t_n$. Maka dengan distribusi binomial
\begin{equation*}
    P(T_n=t_n|\mathbf{x_n},\mathbf{w})=\mathbf{Binomial}(Y_N=y|p,N)=\frac{N!}{(N-y)!y!}p^{y}(1-p)^{N-y}
\end{equation*}
\begin{equation*}
    P(T_n=t_n|\mathbf{x_n},\mathbf{w})=P(T_n=1|x_n,w)^{t_n}\left(1-P(T_n=1|x_n,w)\right)^{1-t_n}
\end{equation*}
\begin{equation}
    P(T_n=t_n|\mathbf{x_n},\mathbf{w})=P_n^{t_n}\left(1-P_n)\right)^{1-t_n} \label{likelihood}
\end{equation}
Berdasarkan data yang diketahui ($\mathbf{x_n}$ dan $t_n$), dengan asumsi seluruh data saling independen, dapat dihitung probability likelihood sebagai
\begin{equation*}
    p(t|\mathbf{X},\mathbf{w})=\prod_{n=1}^{N} P(T_n=t_n|\mathbf{x_n},\mathbf{w})
\end{equation*}
\begin{equation}
    p(t|\mathbf{X},\mathbf{w})=\prod_{n=1}^{N} P_n^{t_n}\left(1-P_n)\right)^{1-t_n}
\end{equation}
Persamaan probabilitas bayes adalah
\begin{equation}
    p(\mathbf{w}|\mathbf{X},t)=\frac{p(t|\mathbf{X},\mathbf{w})p(\mathbf{w})}{p(t|\mathbf{X})}
\end{equation}
Pada kasus anomaly detection, model dilatih untuk mencari parameter $\mathbf{w}$ yang terbaik dalam memetakan seluruh data yang diasumsikan data normal sebagai data yang likely merupakan data normal. Model bayesian memodelkan seluruh variabel sebagai variabel bebas yang terdistribusi dengan suatu probabilitas. Maka, parameter $\mathbf{w}$ yang terbaik adalah ekspektasi dari distribusi probabilitas $p(\mathbf{w}|\mathbf{X},t)$. Sehingga, parameter $\mathbf{w}$ terbaik akan berada pada titik maksimum dari $p(\mathbf{w}|\mathbf{X},t)$. Oleh karena itu, $\mathbf{w}$ terbaik dapat ditentukan dengan menyelesaikan
\begin{equation*}
    \frac{\partial{p(\mathbf{w}|\mathbf{X},t)}}{\partial{\mathbf{w}}} = 0
\end{equation*}
\begin{equation*}
    \frac{\partial}{\partial{\mathbf{w}}} \frac{p(t|\mathbf{X},\mathbf{w})p(\mathbf{w})}{p(t|\mathbf{X})} = 0
\end{equation*}
Karena $p(t|\mathbf{X})$ independen terhadap $\mathbf{w}$, Maka
\begin{equation}
    \frac{\partial}{\partial{\mathbf{w}}} p(t|\mathbf{X},\mathbf{w})p(\mathbf{w}) = 0 \label{ODE}
\end{equation}
Asumsikan parameter $\mathbf{w}$ terdistribusi normal
\begin{equation}
    p(\mathbf{w})=\mathcal{N}_\mathbf{w}\left(0,\sigma^2\mathbf{I}\right)=\mathcal{N}_{w_1}\left(0,\sigma^2\right)\mathcal{N}_{w_2}\left(0,\sigma^2\right) \label{prior}
\end{equation}
Maka, dari persamaan \ref{likelihood} dan \ref{prior}, persamaan \ref{ODE} menjadi
\begin{equation}
    \frac{\partial{}}{\partial{\mathbf{w}}} \left(p(\mathbf{w})\prod_{n=1}^{N} P_n^{t_n}\left(1-P_n\right)^{1-t_n}\right)=0
\end{equation}
Persamaan ini sulit diselesaikan. Namun dengan mengambil logaritma natural kedua sisi akan diperoleh
\begin{equation*}
    \frac{\partial{}}{\partial{\mathbf{w}}} \left(\ln{\left[p(\mathbf{w})\prod_{n=1}^{N} P_n^{t_n}\left(1-P_n\right)^{1-t_n}\right]}\right)=0
\end{equation*}
\begin{equation}
    \frac{\partial{}}{\partial{\mathbf{w}}}\left(\ln{\left[ p(\mathbf{w}) \right]}\right) + \frac{\partial{}}{\partial{w}} \left(\ln{\left[ \prod_{n=1}^{N} P_n^{t_n}\left(1-P_n\right)^{1-t_n}\right] }\right)=0
\end{equation}
Untuk suku pertama:
\begin{equation*}
\ln{\left[ p(\mathbf{w}) \right]} = \ln{\left[ \mathcal{N}_{w_1}\left(0,\sigma^2\right)\mathcal{N}_{w_2}\left(0,\sigma^2\right) \right]}
\end{equation*}
\begin{equation*}
\ln{\left[ p(\mathbf{w}) \right]}=\ln{\left[ \frac{1}{\sqrt{2\pi\sigma^2}}e^{-\frac{w_1^2}{2\sigma^2}} \frac{1}{\sqrt{2\pi\sigma^2}}e^{-\frac{w_2^2}{2\sigma^2}} \right]}
\end{equation*}
\begin{equation*}
\ln{\left[ p(\mathbf{w}) \right]}=-\frac{w_1^2+w_2^2}{2\sigma^2}-\ln\left({2\pi\sigma^2}\right)
\end{equation*}
\begin{equation}
\frac{\partial{}}{\partial{\mathbf{w}}}\left(\ln{\left[ p(\mathbf{w}) \right]}\right)=-\frac{\mathbf{w^T}}{\sigma^2} \label{ODE_1st_term}
\end{equation}

dan untuk suku kedua:
\begin{equation*}
    \ln{\left[ \prod_{n=1}^{N} P_n^{t_n}\left(1-P_n\right)^{1-t_n} \right] }=\sum_{n=1}^{N} \ln{ \left[ P_n^{t_n}\left(1-P_n\right)^{1-t_n} \right] }
\end{equation*}
\begin{equation*}
    \ln{\left[ \prod_{n=1}^{N} P_n^{t_n}\left(1-P_n\right)^{1-t_n} \right] }=\sum_{n=1}^{N} t_n\ln{(P_n)} + (1-t_n)\ln{(1-P_n)}
\end{equation*}
Variabel $t_n$ adalah data yang diketahui dari dataset, ini berarti $t_n$ independen dari $\mathbf{w}$. Sehingga:
\begin{equation}
    \frac{\partial}{\partial{\mathbf{w}}} \ln{\left[ \prod_{n=1}^{N} P_n^{t_n}\left(1-P_n\right)^{1-t_n} \right] } = \sum_{n=1}^{N} t_n \frac{\partial}{\partial{\mathbf{w}}}\ln{(P_n)} + (1-t_n) \frac{\partial}{\partial{\mathbf{w}}} \ln{(1-P_n)} \label{ODE_2nd_term}
\end{equation}
Selesaikan dahulu suku $\frac{\partial{P_n}}{\partial{\mathbf{w}}}$
\begin{equation*}
    \frac{\partial{P_n}}{\partial{\mathbf{w}}} = \frac{\partial{P_n}}{\partial{f}} \frac{\partial{f}}{\partial{\mathbf{w}}} = frac{\partial}{\partial{f}} \left( \frac{1}{1+f} \right) \frac{\partial{f}}{\partial{\mathbf{w}}}
\end{equation*}
Definisikan suatu fungsi $g(f)$ sebagai
\begin{equation}
    g(f) = \frac{\partial}{\partial{f}} \left( \frac{1}{1+f} \right) = -\frac{1}{(1+f)^2} \label{fungsi_g}
\end{equation}
Untuk mempersingkat penulisan, fungsi $g(f)$ akan disingkat menjadi $g$. Diperoleh
\begin{equation}
    \frac{\partial{P_n}}{\partial{\mathbf{w}}} = g \frac{\partial{f}}{\partial{\mathbf{w}}} \label{dPndw_vec}
\end{equation}
dengan
\begin{equation}
    \frac{\partial{f}}{\partial{\mathbf{w}}} = \begin{bmatrix} \frac{\partial{f}}{\partial{w_0}} && \frac{\partial{f}}{\partial{w_1}}\end{bmatrix} \label{dfdw_vec}
\end{equation}
dan dari persamaan \ref{fungsi_f}, dapat diperoleh
\begin{equation}
    \frac{\partial{f}}{\partial{w_n}} = 2(w_n+x_n) \label{dfdwn}
\end{equation}
Maka, dengan persamaan \ref{dPndw_vec} persamaan \ref{ODE_2nd_term} menjadi
\begin{align*}
    \frac{\partial}{\partial{\mathbf{w}}} \ln{\left[ \prod_{n=1}^{N} P_n^{t_n}\left(1-P_n\right)^{1-t_n} \right] } & = \sum_{n=1}^{N} t_n \frac{\partial}{\partial{\mathbf{w}}}\ln{(P_n)} + (1-t_n) \frac{\partial}{\partial{\mathbf{w}}} \ln{(1-P_n)} \\
    & = \sum_{n=1}^{N} t_n \frac{1}{P_n} \frac{\partial{P_n}}{\partial{\mathbf{w}}} - (1-t_n) \frac{1}{(1-P_n)} \frac{\partial{P_n}}{\partial{\mathbf{w}}} \\
    & = \sum_{n=1}^{N} t_n \frac{1}{P_n} g \frac{\partial{f}}{\partial{\mathbf{w}}} - (1-t_n) \frac{1}{(1-P_n)} g \frac{\partial{f}}{\partial{\mathbf{w}}} \\
    & = \sum_{n=1}^{N} \frac{(t_n-P_n)g}{P_n(1-P_n)} \frac{\partial{f}}{\partial{\mathbf{w}}}
\end{align*}
dan dengan \ref{ODE_1st_term} didapatkan persamaan \ref{ODE} menjadi
\begin{equation}
    -\frac{\mathbf{w^T}}{\sigma^2}+\sum_{n=1}^{N} \frac{(t_n-P_n)g}{P_n(1-P_n)} \frac{\partial{f}}{\partial{\mathbf{w}}}=0 \label{ODE_final}
\end{equation}
Persamaan ini tidak dapat diselesaikan secara analitik. Namun dengan metode numerik Newton-Rhapson, solusi $\mathbf{w}$ dapat diperoleh dengan mendefinisikan
\begin{equation}
    \mathbf{F^T}(\mathbf{w}) = -\frac{\mathbf{w^T}}{\sigma^2}+\sum_{n=1}^{N} \frac{(t_n-P_n)g}{P_n(1-P_n)} \frac{\partial{f}}{\partial{\mathbf{w}}} \label{F_vec}
\end{equation}
Maka persamaan \ref{ODE_final} terpenuhi pada kondisi $\mathbf{F^T}(\mathbf{w}) = 0$ yang dapat ditentukan dengan menebak suatu $\mathbf{w}$ awal dan melakukan iterasi
\begin{equation}
    \mathbf{w}_{i+1} = \mathbf{w}_{i} - \mathbf{H}^{-1}\mathbf{F}(\mathbf{w}) \label{iter}
\end{equation}
Dengan $\mathbf{H}$ adalah Hessian matrix
\begin{equation}
    \mathbf{H} = 
    \begin{bmatrix}
    \frac{\partial{F_0}}{\partial{w_0}} & \frac{\partial{F_0}}{\partial{w_1}} \\
    \frac{\partial{F_1}}{\partial{w_0}} & \frac{\partial{F_1}}{\partial{w_1}} 
    \end{bmatrix} \label{H_mat}
\end{equation}
dan $F_0$ serta $F_1$ adalah komponen-komponen dari vektor $\mathbf{F}(\mathbf{w})$. Elemen-elemen pada matrix hessian dapat ditentukan dengan
\begin{equation}
    \frac{\partial{F_m}}{\partial{w_n}} = - \frac{\delta_{mn}}{\sigma^2} + \sum_{n=1}^{N} \left(
    \frac{1}{P_n(1-P_n)}
    \left[
    (t_n-P_n) \frac{\partial{g}}{\partial{w_n}} - g \frac{\partial{P_n}}{\partial{w_n}}
    \right]
    - \frac{(t_n-P_n)g}{P_n^2(1-P_n)^2} (1-2P_n) \frac{\partial{P_n}}{\partial{w_n}}
    \right)
    \frac{\partial{f}}{\partial{w_m}} + \frac{(t_n-P_n)g}{P_n(1-P_n)} \frac{\partial^2{f}}{\partial{w_n}\partial{w_m}} \label{dFn_dwm}
\end{equation}
dengan $\delta_{mn}$ adalah fungsi delta kronecker. Kemudian persamaan \ref{fungsi_g} dan \ref{fungsi_f} dapat diperoleh
\begin{equation}
    \frac{\partial{g}}{\partial{w_n}} = \frac{1}{(1+f)^3} \frac{\partial{f}}{\partial{w_n}} \label{dotg}
\end{equation}
\begin{equation}
    \frac{\partial^2{f}}{\partial{w_n}\partial{w_m}} = 2\delta_{mn} \label{ddotf}
\end{equation}


\subsection{Algorithms}

\begin{python}
    time = datetime.datetime.now()
    print("Start training at : {}".format(time))
    
    w = np.zeros(2)
    var_w = 10
    w_iter = []
    w_iter.append(w)
    maxIter = 10
    nIter = 1
    tol = 10**-6
    delta = 10**6
    while (nIter<maxIter) and (delta>tol):
        Hess = H_mat(t,w,X_scaled)
        F_vector = F_vec(t,w,X_scaled)
        dw = np.matmul(np.linalg.inv(Hess),F_vector)
        delta = sum(dw**2)
        w = w - dw
        print("nIter = %d, delta=%.2e"%(nIter,delta))
        w_iter.append(w)
        nIter += 1
    
    time = datetime.datetime.now()
    print("Done training at : {}".format(time))

\end{python}

    \chapter{Hasil dan Analisis}

\section{Prelude: Pemilihan Training Data}

Perlu diperhatikan bahwa pada kasus ini, anomaly detection dikembangkan dengan pendekatan \emph{unsupervised learning}. Hal ini karena data \texttt{'machine status'} tidak mewakili apakah data berupa anomaly atau tidak.Ingat bahwa permasalahan yang ingin diselesaikan disini adalah mencari pola pada data yang mengindikasi penyebab kerusakan mesin. Hal ini berarti anomaly yang ingin dideteksi terjadi sebelum status mesin \texttt{BROKEN}. 

Data pada bulan 8 dipilih sebagai training data karena pada bulan tersebut tidak terjadi kerusakan mesin. Disini penulis berasumsi bahwa pada bulan 8 tidak terjadi anomali sama sekali pada data, karena tidak terjadi kerusakan mesin. Namun pada bulan lainnya, walau ada bagian data yang memiliki status mesin NORMAL, bisa saja sebenarnya sudah terjadi anomaly yang menyebabkan machine BROKEN pada hari-hari setelahnya.

\section{Hasil}
\subsection{LSTM Autoencoder}

Data dipisah menjadi dua bagian, yaitu data normal yang mengandung data dari status NORMAL dan data anomali yang berasal dari status selain NORMAL (BROKEN dan RECOVERY).

Kemudian data dibagi menjadi 3 dataset, yaitu train, validation, dan test. Data train dan validation diambil pada bulan ke-8 awal dan akhir masing-masing dan digunakan untuk training model, sedangkan data test diambil pada selain bulan ke-8 untuk memprediksi hasil anomali.

Proses training model LSTM Autoencoder dilakukan sebanyak 5 epoch dengan menggunakan GPU (CUDA). Analisis dibagi menjadi dua macam, yaitu dengan PCA dan tanpa PCA.

Kemudian dilakukan prediksi pada data test dari model yang telah dilakukan training sehingga dapat diperoleh nilai loss yang dihasilkan.

    \subsubsection{Dengan PCA}

    \begin{figure}[h]
        \centering
        \includegraphics[width=0.6\textwidth]{resources/LSTM/LSTM_PCA_LossDist.png}
        \caption{Distribusi loss LSTM dengan PCA}
    \end{figure}

    \begin{figure}[h]
        % \centering
        \centerline{\includegraphics[width=1.4\textwidth]{resources/LSTM/LSTM_PCA_model_loss.png}}
        \caption{Plot loss LSTM dengan PCA}
    \end{figure}

    Dengan menggunakan nilai threshold 500, diperoleh jumlah data anomali sebagai berikut.

    \begin{table}[h]
        \centering
        \begin{tabular}{|l|r|r|r|}
            \hline
            \multicolumn{1}{|c|}{\textbf{Jenis anomali}} & \multicolumn{1}{c|}{\textbf{Jumlah}} & \multicolumn{1}{c|}{\textbf{Total data}} & \multicolumn{1}{c|}{\textbf{Persentase (\%)}} \\ \hline
            Anomali pada data NORMAL                     & 33866                                & 160430                                   & 21                                       \\ \hline
            Anomali pada data selain NORMAL              & 2237                                 & 14454                                    & 15                                       \\ \hline
        \end{tabular}
    \end{table}

    Jumlah anomali pada data selain NORMAL tidak mencakup keseluruhan total data sehingga terdapat prediksi yang berada pada data dalam kondisi BROKEN atau RECOVERY.

    \subsubsection{Tanpa PCA}

    \begin{figure}[h]
        \centering
        \includegraphics[width=0.6\textwidth]{resources/LSTM/LSTM_noPCA_LossDist.png}
        \caption{Distribusi loss LSTM tanpa PCA}
    \end{figure}

    \begin{figure}[h]
        % \centering
        \centerline{\includegraphics[width=1.4\textwidth]{resources/LSTM/LSTM_noPCA_model_loss.png}}
        \caption{Plot loss LSTM dengan PCA}
    \end{figure}

    Dengan menggunakan nilai threshold 3500, diperoleh jumlah data anomali sebagai berikut.

    \begin{table}[h]
        \centering
        \begin{tabular}{|l|r|r|r|}
            \hline
            \multicolumn{1}{|c|}{\textbf{Jenis anomali}} & \multicolumn{1}{c|}{\textbf{Jumlah}} & \multicolumn{1}{c|}{\textbf{Total data}} & \multicolumn{1}{c|}{\textbf{Persentase (\%)}} \\ \hline
            Anomali pada data NORMAL                     & 30182                                & 160430                                   & 19                                       \\ \hline
            Anomali pada data selain NORMAL              & 4979                                 & 14454                                    & 34                                       \\ \hline
        \end{tabular}
    \end{table}

    Terlihat bahwa jumlah anomali pada data NORMAL lebih sedikit 3\% dari analisis dengan PCA, Namun jumlah anomali pada data selain NORMAL juga meningkat hampir 2 kali lipat.

\subsection{Bayesian Probability}

Data dipisah menjadi training data dan test data. Tidak ada validation data disini, karena validation data digunakan untuk mencegah model overfitting pada training data. Namun karena model bayesian adalah distribusi probabilitas, maka validation data akan menggeser kontur probabilitas dari overfitting pada training data menjadi tepat fit pada training dan validation data sekaligus. Jadi hasilnya tidak akan ada bedanya jika model dilatih pada training dan validation data yang digabungkan. Karena itu, validation data sudah digabungkan kedalam training data, yaitu data pada bulan 8.

Hasil prediksi model Bayesian pada test data menghasilkan distribusi probabilitas data merupakan data normal sebagai berikut.
\begin{figure}[h]
    \centering
    \includegraphics[width=0.8\textwidth]{resources/Bayes/Bayes_ProbDist.png}
    \caption{Prediksi Probabilitas bukan Anomali}
\end{figure}
Kemudian probabilitas tidak terjadi anomali dan terjadi anomali pada tiap waktu sebagai berikut.
\begin{figure}[h]
    %\centering
    \centerline{\includegraphics[width=1.4\textwidth]{resources/Bayes/Bayes_normal_PMF.png}}
    \caption{Prediksi Probabilitas bukan Anomali}
\end{figure}
\begin{figure}[h]
    %\centering
    \centerline{\includegraphics[width=1.4\textwidth]{resources/Bayes/Bayes_anomaly_PMF.png}}
    \caption{Prediksi Probabilitas terjadi Anomali}
\end{figure}
Dengan menetapkan nilai threshold sebesar 0.85, ini berarti jika probabilitas suatu data merupakan data normal dibawah 0.85 sudah ditetapkan sebagai data yang likely merupakan anomaly. Diperoleh jumlah data anomali sebagai berikut.
\begin{table}[h]
    \centering
    \begin{tabular}{|l|r|r|r|}
        \hline
        \multicolumn{1}{|c|}{\textbf{Jenis anomali}} & \multicolumn{1}{c|}{\textbf{Jumlah}} & \multicolumn{1}{c|}{\textbf{Total data}} & \multicolumn{1}{c|}{\textbf{Persentase (\%)}} \\ \hline
        Anomali pada data NORMAL                     & 34527                                & 160430                                   & 22                                       \\ \hline
        Anomali pada data selain NORMAL              & 2567                                 & 14454                                    & 18                                       \\ \hline
    \end{tabular}
\end{table}

\section{Analisis}
\subsection{Komparasi Kecepatan Proses}
Kecepatan proses yang diukur adalah durasi waktu yang dibutuhkan untuk model training. Didapatkan hasil berikut.

\begin{table}[h]
    \centering
    \begin{tabular}{|l|r|l|}
        \hline
        \multicolumn{1}{|c}{\textbf{Model}} & \multicolumn{1}{|c|}{\textbf{Durasi Model Training}} & \multicolumn{1}{c|}{\textbf{Keterangan}} \\ \hline
        Interquartile Range (IQR)   &        1s 100ms 176$\mu$s    & Acuan \\
        K-Means Clustering          &       46s 063ms 282$\mu$s    & Acuan \\
        Isolation Forest            &       15s 139ms 811$\mu$s    & Acuan \\
        Bayesian Probability        &       15s 031ms 475$\mu$s    & Kembangan \\
        LSTM Autoencoder dengan PCA &   18m 02s 607ms 847$\mu$s    & Kembangan \\
        LSTM Autoencoder tanpa PCA  &   39m 20s 885ms 569$\mu$s    & Kembangan \\
        \hline
    \end{tabular}
\end{table}

Terlihat bahwa durasi model training cukup cepat, kecuali untuk model LSTM Autoencoder. Hal ini karena LSTM Autoencoder memiliki basis Neural Network yang mempunyai multilayer. Dalam model yang penulis kembangkan, total layer sebanyak 128 lapis, yang membuat proses komputasi cukup ekstensif walau sudah dibantu dengan menggunakan driver CUDA.

Model IQR dan Bayesian membutuhkan durasi yang cukup sedikit. Hal ini karena model dilatih dengan perhitungan statistik yang sederhana secara komputasi. Model IQR hanya perlu menghitung tiap data terhadap range antar quartil atas dan quartil bawah. Kemudian model Bayesian sudah menggunakan penyederhanaan matematik secara analitik yang cukup panjang, sehingga iterasi numerik yang perlu dilakukan tidak terlalu kompleks. Hal ini juga dibantu dengan scale down data input pada model Bayesian. Walaupun model Bayesian yang menggunakan matematik yang jauh lebih rumit dibandingkan IQR membuat durasi training 15x lebih lama.

Model K-Means Clustering dan Isolation Forest memakan durasi yang kecil karena pendekatan yang dilakukan adalah menganggap data hanya terbagi antara 2 kluster saja, yaitu kluster normal dan anomaly. Apabila asumsi kluster lebih dari 2 mungkin akan memakan waktu lebih lama, namun ini sudah bukan lagi Anomaly Detection melainkan Clustering Data.

\subsection{Perbandingan Prediksi Anomali}
Ketiga model memiliki ....

    \begin{figure}[h]
        %\centering
        \centerline{\includegraphics[width=1.4\textwidth]{resources/LSTM/LSTM_noPCA_sensor_11.png}}
        \caption{Hasil deteksi anomali model LSTM tanpa PCA}
    \end{figure}
    
    \begin{figure}[h]
        %\centering
        \centerline{\includegraphics[width=1.4\textwidth]{resources/LSTM/LSTM_PCA_sensor_11.png}}
        \caption{Hasil deteksi anomali model LSTM dengan PCA}
    \end{figure}
    
    \begin{figure}[h]
        %\centering
        \centerline{\includegraphics[width=1.4\textwidth]{resources/Bayes/Bayes_sensor_11.png}}
        \caption{Hasil deteksi anomali model Bayesian}
    \end{figure}

    Karena prediksi pada model acuan tidak menghasilkan distribusi loss dari model maka akan dianalisis berdasarkan jumlah anomali yang diprediksi.

    Hasil prediksi model kembangan seluruhnya menghasilkan jumlah yang lebih besar dari model acuan. Model LSTM dengan dan tanpa PCA beserta Bayesian masing-masing memprediksi anomali sebanyak 21\%, 19\%, dan 22\%, sedangkan model IQR, K-Means Clustering, dan Isolation Forest masing-masing menghasilkan 14\%, 13\%, dan 13\%. Namun perbandingan jumlah prediksi dirasa kurang akurat karena bergantung oleh nilai threshold yang digunakan tiap model.

    \chapter{Penutup}

\section{Kesimpulan}

\section{Saran}

    %----------------------------------------------------------------%

    % Daftar pustaka
    \printbibliography[heading=bibintoc,title={Daftar Pustaka}]

    % Index
    %\appendix

    \part*{Lampiran}
    \addcontentsline{toc}{part}{Lampiran}
    

    Daftar file jupyter notebook yang penulis lampirkan dalam link google drive berikut:
    https://drive.google.com/drive/folders/1mjf-hefWUkvGAlhulI-Cl-TQ22Q1rUwh?usp=sharing

    Jupyter notebook program acuan dan modifikasi:
    \begin{enumerate}[label=\textbf{L.\arabic*}]
        \item Time Series Anomaly Detection -  Steps 1 and 2.ipynb
        \item Time Series Anomaly Detection - Steps 3 to 5.ipynb
        \item Model\textunderscore Reference\textunderscore Result\textunderscore Analysis.ipynb (modifikasi penulis) \label{refmodified_ipynb}
    \end{enumerate}

    Jupyter notebook program kembangan:
    \begin{enumerate}[label=\textbf{L.\arabic*}]
        \setcounter{enumi}{4}
        \item LSTM\textunderscore ver3\textunderscore revB.ipynb \label{lstm_pca_ipynb}
        \item LSTM\textunderscore ver3\textunderscore revB\textunderscore Result\textunderscore Analysis.ipynb \label{lstm_pca_res_ipynb}
        \item LSTM\textunderscore noPCA.ipynb \label{lstm_nopca_ipynb}
        \item LSTM\textunderscore noPCA\textunderscore Result\textunderscore Analysis.ipynb \label{lstm_pca_res_ipynb}
        \item Bayes\textunderscore revB.ipynb \label{bayes_ipynb}
        \item Bayes\textunderscore Result\textunderscore Analysis.ipynb \label{lstm_pca_res_ipynb}
    \end{enumerate}

    Beberapa grafik yang representatif. Lengkapnya ada pada link google drive diatas.

    \begin{figure}[h]
        %\centering
        \centerline{\includegraphics[width=1.4\textwidth]{resources/Acuan/IQR_sensor_11.png}}
        \caption{Hasil prediksi anomali model IQR} \label{IQR11}
    \end{figure}
    \begin{figure}[h]
        %\centering
        \centerline{\includegraphics[width=1.4\textwidth]{resources/Acuan/KMeans_sensor_11.png}}
        \caption{Hasil prediksi anomali model K-Means Clustering} \label{KM11}
    \end{figure}
    \begin{figure}[h]
        %\centering
        \centerline{\includegraphics[width=1.4\textwidth]{resources/Acuan/IsoFor_sensor_11.png}}
        \caption{Hasil prediksi anomali model Isolation Forest} \label{IF11}
    \end{figure}
    \begin{figure}[h]
        %\centering
        \centerline{\includegraphics[width=1.4\textwidth]{resources/Acuan/IQR_machine_status.png}}
        \caption{Anomali model IQR dan status mesin} \label{IQRms}
    \end{figure}
    \begin{figure}[h]
        %\centering
        \centerline{\includegraphics[width=1.4\textwidth]{resources/Acuan/KMeans_machine_status.png}}
        \caption{Anomali model K-Means Clustering dan status mesin} \label{KMms}
    \end{figure}
    \begin{figure}[h]
        %\centering
        \centerline{\includegraphics[width=1.4\textwidth]{resources/Acuan/IsoFor_machine_status.png}}
        \caption{Anomali model Isolation Forest dan status mesin} \label{IFms}
    \end{figure}
    \begin{figure}[h]
        %\centering
        \centerline{\includegraphics[width=1.4\textwidth]{resources/Bayes/Bayes_machine_status.png}}
        \caption{Anomali model Bayesian dan status mesin} \label{Bms}
    \end{figure}
    \begin{figure}[h]
        %\centering
        \centerline{\includegraphics[width=1.4\textwidth]{resources/LSTM/LSTM_noPCA_machine_status.png}}
        \caption{Anomali model LSTM tanpa PCA dan status mesin} \label{nPms}
    \end{figure}
    \begin{figure}[h]
        %\centering
        \centerline{\includegraphics[width=1.4\textwidth]{resources/LSTM/LSTM_PCA_machine_status.png}}
        \caption{Anomali model LSTM dengan PCA dan status mesin} \label{wPms}
    \end{figure}


    %\chapter{Notebook}

Daftar file Jupyter Notebook yang penulis lampirkan terdapat pada link Google Drive berikut: \\
\url{https://drive.google.com/drive/folders/1mjf-hefWUkvGAlhulI-Cl-TQ22Q1rUwh}

Jupyter Notebook program acuan dan modifikasi:
\begin{enumerate}[label=\textbf{L.\arabic*}]
    \item \texttt{Time Series Anomaly Detection -  Steps 1 and 2.ipynb}
    \item \texttt{Time Series Anomaly Detection - Steps 3 to 5.ipynb}
    \item \texttt{Model\textunderscore Reference\textunderscore Result\textunderscore Analysis.ipynb} (modifikasi penulis) \label{refmodified_ipynb}
\end{enumerate}

Jupyter Notebook program kembangan:
\begin{enumerate}[label=\textbf{L.\arabic*}]
    \setcounter{enumi}{4}
    \item \texttt{LSTM\textunderscore ver3\textunderscore revB.ipynb} \label{lstm_pca_ipynb}
    \item \texttt{LSTM\textunderscore ver3\textunderscore revB\textunderscore Result\textunderscore Analysis.ipynb} \label{lstm_pca_res_ipynb}
    \item \texttt{LSTM\textunderscore noPCA.ipynb} \label{lstm_nopca_ipynb}
    \item \texttt{LSTM\textunderscore noPCA\textunderscore Result\textunderscore Analysis.ipynb} \label{lstm_pca_res_ipynb}
    \item \texttt{Bayes\textunderscore revB.ipynb} \label{bayes_ipynb}
    \item \texttt{Bayes\textunderscore Result\textunderscore Analysis.ipynb} \label{lstm_pca_res_ipynb}
\end{enumerate}

    %\chapter{Grafik}

Beberapa grafik yang representatif. Lengkapnya terdapat pada link Google Drive pada lampiran sebelumnya.

\begin{figure}[h]
    %\centering
    \centerline{\includegraphics[width=1.4\textwidth]{resources/Acuan/IQR_sensor_11.png}}
    \caption{Hasil prediksi anomali model IQR} \label{IQR11}
\end{figure}
\begin{figure}[h]
    %\centering
    \centerline{\includegraphics[width=1.4\textwidth]{resources/Acuan/KMeans_sensor_11.png}}
    \caption{Hasil prediksi anomali model K-Means Clustering} \label{KM11}
\end{figure}
\begin{figure}[h]
    %\centering
    \centerline{\includegraphics[width=1.4\textwidth]{resources/Acuan/IsoFor_sensor_11.png}}
    \caption{Hasil prediksi anomali model Isolation Forest} \label{IF11}
\end{figure}
\begin{figure}[h]
    %\centering
    \centerline{\includegraphics[width=1.4\textwidth]{resources/Acuan/IQR_machine_status.png}}
    \caption{Anomali model IQR dan status mesin} \label{IQRms}
\end{figure}
\begin{figure}[h]
    %\centering
    \centerline{\includegraphics[width=1.4\textwidth]{resources/Acuan/KMeans_machine_status.png}}
    \caption{Anomali model K-Means Clustering dan status mesin} \label{KMms}
\end{figure}
\begin{figure}[h]
    %\centering
    \centerline{\includegraphics[width=1.4\textwidth]{resources/Acuan/IsoFor_machine_status.png}}
    \caption{Anomali model Isolation Forest dan status mesin} \label{IFms}
\end{figure}
\begin{figure}[h]
    %\centering
    \centerline{\includegraphics[width=1.4\textwidth]{resources/Bayes/Bayes_machine_status.png}}
    \caption{Anomali model Bayesian dan status mesin} \label{Bms}
\end{figure}
\begin{figure}[h]
    %\centering
    \centerline{\includegraphics[width=1.4\textwidth]{resources/LSTM/LSTM_noPCA_machine_status.png}}
    \caption{Anomali model LSTM tanpa PCA dan status mesin} \label{nPms}
\end{figure}
\begin{figure}[h]
    %\centering
    \centerline{\includegraphics[width=1.4\textwidth]{resources/LSTM/LSTM_PCA_machine_status.png}}
    \caption{Anomali model LSTM dengan PCA dan status mesin} \label{wPms}
\end{figure}

\end{document}

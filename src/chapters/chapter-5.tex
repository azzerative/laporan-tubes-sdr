\chapter{Penutup}

\section{Kesimpulan}
Berdasarkan analisis pada hasil prediksi anomaly keenam model seperti pada \ref{ANALISIS}, disimpulkan:
\begin{enumerate}
    \item Model kembangan unggul dalam mendeteksi kerusakan pada bulan-bulan awal, namun model acuan unggul pada bulan-bulan akhir.
    \item Model kembangan LSTM Autoencoder berhasil mendeteksi anomali yang overlap dengan data normal.
    \item Model terbaik adalah LSTM Autoencoder dengan PCA dalam kriteria keberhasilan memprediksi kerusakan mesin dan \emph{false positive rate}.
\end{enumerate}

\section{Saran}

Masih banyak metode dan model lain yang dapat digunakan dalam kasus anomaly detection. Tidak hanya itu saja, tahap preprocessing data juga dapat berpenguh pada performa model. Kemudian, banyak faktor-faktor luar yang tidak terkandung dalam data yang dapat mempengaruhi \emph{model learning}, contohnya maintenance mesin tak terjadwal yang mungkin dilakukan oleh pemilik mesin. Durasi maintenance mesin yang berubah-ubah juga dapat membuat model sulit mengenali anomali. Kerusakan yang tidak diakibatkan oleh kesalahan internal mesin juga mungkin terjadi, misalnya kerusakan mesin akibat \emph{human error}, ini akan menyebabkan kerusakan mendadak tanpa terdeteksi anomali sebelum kerusakan terjadi.

Dengan begitu, tidak ada suatu metode eksak yang terbaik dalam anomaly detection. Yang ada hanyalah model yang paling cocok pada suatu mesin untuk memprediksi kerusakan. Seberapa \emph{severe} kerusakan yang dapat terjadi juga menjadi pertimbangan, serta \emph{computational cost} yang dibutuhkan untuk melatih model.